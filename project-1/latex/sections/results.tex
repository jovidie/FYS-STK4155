%================================================================
\section{Results}\label{sec:results}
% Present results and give a critical discussion of my work in 
% the context of other work. Relate the work to previous studies, 
% and make sure the results are reproducible. Include information 
% in figure captions such that they can give the reader enough to 
% understand the main gist of the report.
%================================================================
\subsection{Synthetic data analysis}\label{ssec:synthetic_data}
I generated two synthetic data set, one of which included stochastic noise, from the Franke function in Equation \eqref{eq:franke_function}. To keep the initial analysis efficient and interpretable, I used $50 \times 50$ data points and noise $\epsilon \sim \mathcal{N}(0, 0.1)$ with less variance than suggested in the project description. 

\subsubsection{Ordinary Least Squares}\label{sssec:ols}
I performed an OLS regression analysis up to a fifth polynomial degree, and computed the MSE and R$^{2}$-score for performance on both data set. The model's coefficients are shown in Figure \ref{fig:ols_beta_smooth} and Figure \ref{fig:ols_beta}. The MSE and R$^{2}$-score is shown in Figure \ref{fig:ols_error_smooth} and Figure \ref{fig:ols_error}.
\begin{figure}
    \centering
    \includegraphics[width=0.9\linewidth]{project-1/latex/figures/ols_beta_smooth_N50.pdf}
    \caption{The figure shows $\beta$ values for the OLS model, with features of polynomial degree up to fifth order. The index $i$ denotes the index of $\beta_{i}$.The model was trained and tested on synthetic data without stochastic noise.}
    \label{fig:ols_beta_smooth}
\end{figure}
\begin{figure}
    \centering
    \includegraphics[width=0.9\linewidth]{project-1/latex/figures/ols_beta_N50.pdf}
    \caption{The figure shows $\beta$ values for the OLS model, with features of polynomial degree up to fifth order. The index $i$ denotes the index of $\beta_{i}$. The model was trained and tested on synthetic data which included stochastic noise $\mathcal{N}(0, 0.1)$.}
    \label{fig:ols_beta_smooth}
\end{figure}
\begin{figure}
    \centering
    \includegraphics[width=0.9\linewidth]{project-1/latex/figures/ols_error_smooth_N50.pdf}
    \caption{MSE and R$^{2}$-score computed from predictions on test data, as a function of the polynomial degree of the input features. The model was trained and tested on synthetic data without stochastic noise.}
    \label{fig:ols_error_smooth}
\end{figure}
\begin{figure}
    \centering
    \includegraphics[width=0.9\linewidth]{project-1/latex/figures/ols_error_N50.pdf}
    \caption{MSE and R$^{2}$-score computed on test data, as a function of the polynomial degree of the input features. The model was trained and tested on synthetic data which included stochastic noise $\mathcal{N}(0, 0.1)$.}
    \label{fig:ols_error}
\end{figure}
In addition, I performed the analysis with feature scaling, the resulting figures can be found in \ref{ap:additional_analysis}.

When increasing the order of polynomial features, the coefficient $\beta$ increase in value as well as variance. The error decrease with the increase of polynomial order, suggesting the model is better able to explain the complexity of the data. When comparing the result in Figure \ref{fig:ols_error_smooth} and Figure \ref{fig:ols_error}, the model performs better on data which does not include noise. 


%------------ Hastie -----------------------------


%------------ Bootstrap --------------------------


%------------ Crossval ---------------------------


\subsubsection{Ridge}\label{sssec:ridge_synthetic}



\subsubsection{Lasso}\label{sssec:lasso_synthetic}








When noise is included in the function data, the error increases for all polynomial degrees. Since the R$^{2}$-score determines how large a proportion of the variance in the dependent variable is explained by the independent variable, it seems the model only explains the input features.

Looking at $\beta$-values for the model fitting the function data without noise in Figure \ref{fig:ols_beta}, and the function data with noise Figure \ref{fig:ols_beta_noisy}. As the polynomial degree increase, the absolute values of $\beta$ increase. However, both the absolute values and variance between each $\beta_{i}$ increase when noise is included in the function.  
\begin{figure}
    \centering
    \includegraphics[width=0.9\linewidth]{project-1/latex/figures/ols_beta.pdf}
    \caption{The figure shows beta values for different polynomial degree.}
    \label{fig:ols_beta}
\end{figure}

\begin{figure}
    \centering
    \includegraphics[width=0.9\linewidth]{project-1/latex/figures/ols_beta_noisy.pdf}
    \caption{The figure shows beta values for different polynomial degree, when the function includes noise.}
    \label{fig:ols_beta_noisy}
\end{figure}

I continued the regression analysis with data from the Franke function which included noise.