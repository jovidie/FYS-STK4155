%==============================================================
\section{Methods}\label{sec:methods}
% Describe the methods and algorithms used, include any formulas. 
% Explain how everything is implemented, and possibly mention the structure of the algorithm. Add demonstrations such as tests, selected runs and validations. 
%==============================================================
%------------ Theory ------------------------------------------
In this project I have studied regression methods to predict terrain, when the model was trained on synthetic data.  
- Linear regression
%------------ Data --------------------------------------------
\subsection{Data}\label{ssec:data}
To implement the regression methods, I used synthetic data generated from the two-dimensional Franke function \eqref{eq:franke_function}, commonly used as a test function in polynomial fitting. The function was used to generate data, which was used to train and test the models. 
\begin{equation}\label{eq:franke_function}
\begin{split}
    f(x, y) &= \frac{3}{4} \exp(- \frac{(9x-2)^{2}}{4} - \frac{(9y-2)^{2}}{4} ) \\
    &+ \frac{3}{4} \exp(- \frac{(9x-2)^{2}}{4} - \frac{(9y-2)^{2}}{4} ) \\
    &+ \frac{3}{4} \exp(- \frac{(9x+1)^{2}}{49} - \frac{(9y+1)}{10} ) \\ 
    &- \frac{1}{5} \exp(- (9x-4)^{2} - (9y-7)^{2} ) 
\end{split}, 
\end{equation}
where $x, y \in [0, 1]$. I generated data points from a uniform distribution, to ensure a good representation of the domain. The data was not scaled as the input to the function was scaled, resulting in the output being scaled to that domain.

In addition to synthetic data, I tested the models on real terrain data. The real data was generated using the EarthExplorer \cite{usgov:2024:earthexplorer}, with selected World Features. To study the performance of the models on various terrain, I generated data set which included different features. Information on the terrain data can be found in Table \ref{tab:terrain_data}, in Appendix \ref{ap:terrain_data}.

The terrain data was not preprocessed, which required it to be scaled. I started with standard scaling, and also tested min-max scaling. However, these did not work for only scaling the terrain data values, I standardized the terrain data by subtracting the mean value, and dividing by the standard deviation.

Both the function data and the terrain data was split up in train and test set.

%------------ Regression methods ------------------------------
\subsection{Regression methods}\label{ssec:regression_methods}
For problems of a continuous nature, linear regression methods are commonly used, as they provide analytical expressions which can provide statistical insight. 

The general aim of regression methods are to find a function which describes the data. That is, given a set of data $\mathbf{y}$, the aim is to find a continuous function $f$ which model the data as
\begin{equation*}
    \mathbf{y} = f( \mathbf{x} ) + \mathbf{\epsilon}, 
\end{equation*}
where $\mathbf{\epsilon} \in \mathcal{N}(0, \sigma^{2})$. In this project I will apply linear regression to find an approximation of the data 
\begin{equation*}
    \mathbf{\Tilde{y}} = \mathbf{X} \mathbf{\beta},
\end{equation*}
where $\mathbf{\beta}$ are the optimal parameters. More specifically, I want to find the relationship between the input features and the target, by determining the coefficients. 

To measure the quality of the model, I define a loss function 
\begin{equation*}
    C (\mathbf{\beta}) = \frac{1}{n} \big( (\mathbf{y} - \mathbf{X} \mathbf{\beta})^{T} (\mathbf{y} - \mathbf{X} \mathbf{\beta}) \big)
\end{equation*}

- approximation by beta
- cost function
I set up design matrices to a range of polynomial degrees, and fit the OLS model using function data. 
I implemented the OLS method and fit the function data, 

- Ridge
- Lasso

%------------ Resampling methods -------------------------------------------
\subsection{Resampling techniques}\label{ssec:resampling_techniques}
- Bootstrap
- Cross-validation

%------------ Model evaluation --------------------------------
\subsection{Model evaluation}\label{ssec:evaluation}
To evaluate the performance of each model, I used both the mean squared error (MSE) and the R$^{2}$-score
- Error analysis
- Bias-variance trade off

%------------ Tools -------------------------------------------
\subsection{Tools}\label{ssec:tools}
All code was implemented in \verb|Python|. In addition, I used utility functions from \verb|scikit-learn| for preprocessing data, evaluating models, and the Lasso regression method. To produce and stylize figures I used the \verb|matplotlib| and \verb|seaborn| libraries, respectively.

In order to explore the data and function during implementation, I used notebooks, as well as unit tests in \verb|test.py|.