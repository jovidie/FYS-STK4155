%================================================================
%------------------------- Abstract -----------------------------
%================================================================
% You summarize your work short and sweet, and reveal very sensible findings. Good! For an abstract, it is customary to include some stage setting at the beginning. The first sentence should reveal what field we are in. The next few might narrow in on which specific subfield we are in. Then, you explain your motivation, or present the "gap" in knowledge that your research will fill.

% Of course, it's not easy to make a motivation for this project, because it is just a project! But I encourage you to attempt to create a motivation that corresponds to the results you want to highlight here. For example, maybe you see a need to explore when to use OLS versus Ridge, because OLS performs better than Ridge unless there's overfitting. It's up to you :) 


% The abstract can then conclude with the broader impact of your findings. What does it mean for the scientific community? Or the general public?
\begin{abstract}
    In this project, I have studied three regression methods, Ordinary Least Squares (OLS), Ridge and Lasso, and compared their performance on synthetic and real terrain data. I found that the choice of method relies on the problem to be solved, and that the OLS method is sufficient when using noise free synthetic data from the Franke function. When introducing noise to the function, Ridge regression performed better as it did not overfit the data. I also performed a bias-variance trade-off analysis, using resampling methods such as bootstrap and cross-validation, and found the optimal polynomial degree for the input data.
\end{abstract}