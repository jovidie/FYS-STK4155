%================================================================
\section{Introduction}\label{sec:introduction}
% Motivate the reader and present overarching ideas, and 
% background on the subject of the project. Mention what I have 
% done and present the structure of the report, that is how it is 
% organized.
%================================================================
% The stage-setting is great, good use of a reference! It would be great to have the set-up lead into what your motivation is for this project, as discussed above. Then, you can follow with your good summary of what was done, but include the why it was done as well. Finally, you can conclude briefly with your results, just like in the intro. Very good that you outline the sections! 

Artificial Intelligence (AI) and Machine Learning (ML) methods have many areas of applications. It allows us to automate manual processes, make sense of complex data, and learn patterns not readily available to humans. 

% Alternative 1
Today, the use of ML methods can be found in most industries, such as fraud detection, diagnosing disease, and increase efficiency in manufacturing processes \cite{forbes:2023:machine_learning}. As AI and ML models are becoming more advanced, they require more data, and the energy consumption of training them increase. However, these resources can could also impact the environment in a negative manner, and it is important to think of the complexity of a model compared to the complexity of the problem. Complex is not always better! 

% Alternative 2
Today, climate change is of great concern, and one effect is the increase in more extreme weather. A consequence of this is floods and the damaging effect it can have on both man-made and natural structures. One way to detect and possible prevent the destruction is to use ML to predict areas of increased risk, based on weather features \cite{supriya:2015:regression_analysis}. In combination with a model of the terrain \cite{opengeohub:2020:eu_terrain}.

In this project I will focus on Linear Regression, and study the effect on model perfomance when Ridge and Lasso regularization is included. In addition, I will compare the predictive performance on synthetic data from the Franke function, with more complex data of real terrain. To analyse the models in more detail, I will use resampling methods such as bootstrap and cross-validation, and study the bias-variance trade-off when increasing feature complexity.

In Section II, I will present the methods and tools used in this project. Continuing with Section III, where I show the results and discuss my findings. In Section IV, I conclude and discuss possible future work.


 
Terrain modeling \cite{opengeohub:2020:eu_terrain}

In this project I will focus on Linear Regression ML methods, and their use in modeling and analysing terrain data. More specifically, I will study the difference in performance of Ordinary Least Squares, Ridge, and Lasso regression method, on synthetic terrain data generated using the Franke function. In addition, I will compare the model's performance when testing on real terrain data. To analyse the models in more detail, I will use resampling methods such as bootstrap and cross-validation, and study the bias-variance trade-off.

