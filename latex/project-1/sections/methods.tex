%================================================================
\section{Methods}\label{sec:methods}
% Describe the methods and algorithms used, include any formulas. 
% Explain how everything is implemented, and possibly mention the 
% structure of the algorithm. Add demonstrations such as tests, 
%  selected runs and validations. 
%================================================================

\subsection{Data}\label{ssec:data}
To train the model we need data which resemble our problem. The Franke function \eqref{eq:franke_function} is used as a test function in problems dealing with interpolation, and will be used to generate dataset. 

\begin{equation}\label{eq:franke_function}
\begin{split}
    f(x, y) &= \frac{3}{4} \exp(- \frac{(9x-2)^{2}}{4} - \frac{(9y-2)^{2}}{4} ) \\
    &+ \frac{3}{4} \exp(- \frac{(9x-2)^{2}}{4} - \frac{(9y-2)^{2}}{4} ) \\
    &+ \frac{3}{4} \exp(- \frac{(9x+1)^{2}}{49} - \frac{(9y+1)}{10} ) \\ 
    &- \frac{1}{5} \exp(- (9x-4)^{2} - (9y-7)^{2} ) 
\end{split}
\end{equation}
- Topographic data
- Preprocessing


\subsection{Regression models}\label{ssec:regression_models}
- OLS
- Ridge
- Lasso

\subsection{Resampling techniques}\label{ssec:resampling_techniques}
- Bootstrap
- Cross-validation

\subsection{Model evaluation}\label{ssec:evaluation}
- Bias-variance trade off


\subsection{Tools}\label{ssec:tools}